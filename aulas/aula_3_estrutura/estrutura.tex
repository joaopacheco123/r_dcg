% Options for packages loaded elsewhere
\PassOptionsToPackage{unicode}{hyperref}
\PassOptionsToPackage{hyphens}{url}
\PassOptionsToPackage{dvipsnames,svgnames,x11names}{xcolor}
%
\documentclass[
  letterpaper,
  DIV=11,
  numbers=noendperiod]{scrartcl}

\usepackage{amsmath,amssymb}
\usepackage{iftex}
\ifPDFTeX
  \usepackage[T1]{fontenc}
  \usepackage[utf8]{inputenc}
  \usepackage{textcomp} % provide euro and other symbols
\else % if luatex or xetex
  \usepackage{unicode-math}
  \defaultfontfeatures{Scale=MatchLowercase}
  \defaultfontfeatures[\rmfamily]{Ligatures=TeX,Scale=1}
\fi
\usepackage{lmodern}
\ifPDFTeX\else  
    % xetex/luatex font selection
\fi
% Use upquote if available, for straight quotes in verbatim environments
\IfFileExists{upquote.sty}{\usepackage{upquote}}{}
\IfFileExists{microtype.sty}{% use microtype if available
  \usepackage[]{microtype}
  \UseMicrotypeSet[protrusion]{basicmath} % disable protrusion for tt fonts
}{}
\makeatletter
\@ifundefined{KOMAClassName}{% if non-KOMA class
  \IfFileExists{parskip.sty}{%
    \usepackage{parskip}
  }{% else
    \setlength{\parindent}{0pt}
    \setlength{\parskip}{6pt plus 2pt minus 1pt}}
}{% if KOMA class
  \KOMAoptions{parskip=half}}
\makeatother
\usepackage{xcolor}
\setlength{\emergencystretch}{3em} % prevent overfull lines
\setcounter{secnumdepth}{-\maxdimen} % remove section numbering
% Make \paragraph and \subparagraph free-standing
\ifx\paragraph\undefined\else
  \let\oldparagraph\paragraph
  \renewcommand{\paragraph}[1]{\oldparagraph{#1}\mbox{}}
\fi
\ifx\subparagraph\undefined\else
  \let\oldsubparagraph\subparagraph
  \renewcommand{\subparagraph}[1]{\oldsubparagraph{#1}\mbox{}}
\fi

\usepackage{color}
\usepackage{fancyvrb}
\newcommand{\VerbBar}{|}
\newcommand{\VERB}{\Verb[commandchars=\\\{\}]}
\DefineVerbatimEnvironment{Highlighting}{Verbatim}{commandchars=\\\{\}}
% Add ',fontsize=\small' for more characters per line
\usepackage{framed}
\definecolor{shadecolor}{RGB}{241,243,245}
\newenvironment{Shaded}{\begin{snugshade}}{\end{snugshade}}
\newcommand{\AlertTok}[1]{\textcolor[rgb]{0.68,0.00,0.00}{#1}}
\newcommand{\AnnotationTok}[1]{\textcolor[rgb]{0.37,0.37,0.37}{#1}}
\newcommand{\AttributeTok}[1]{\textcolor[rgb]{0.40,0.45,0.13}{#1}}
\newcommand{\BaseNTok}[1]{\textcolor[rgb]{0.68,0.00,0.00}{#1}}
\newcommand{\BuiltInTok}[1]{\textcolor[rgb]{0.00,0.23,0.31}{#1}}
\newcommand{\CharTok}[1]{\textcolor[rgb]{0.13,0.47,0.30}{#1}}
\newcommand{\CommentTok}[1]{\textcolor[rgb]{0.37,0.37,0.37}{#1}}
\newcommand{\CommentVarTok}[1]{\textcolor[rgb]{0.37,0.37,0.37}{\textit{#1}}}
\newcommand{\ConstantTok}[1]{\textcolor[rgb]{0.56,0.35,0.01}{#1}}
\newcommand{\ControlFlowTok}[1]{\textcolor[rgb]{0.00,0.23,0.31}{#1}}
\newcommand{\DataTypeTok}[1]{\textcolor[rgb]{0.68,0.00,0.00}{#1}}
\newcommand{\DecValTok}[1]{\textcolor[rgb]{0.68,0.00,0.00}{#1}}
\newcommand{\DocumentationTok}[1]{\textcolor[rgb]{0.37,0.37,0.37}{\textit{#1}}}
\newcommand{\ErrorTok}[1]{\textcolor[rgb]{0.68,0.00,0.00}{#1}}
\newcommand{\ExtensionTok}[1]{\textcolor[rgb]{0.00,0.23,0.31}{#1}}
\newcommand{\FloatTok}[1]{\textcolor[rgb]{0.68,0.00,0.00}{#1}}
\newcommand{\FunctionTok}[1]{\textcolor[rgb]{0.28,0.35,0.67}{#1}}
\newcommand{\ImportTok}[1]{\textcolor[rgb]{0.00,0.46,0.62}{#1}}
\newcommand{\InformationTok}[1]{\textcolor[rgb]{0.37,0.37,0.37}{#1}}
\newcommand{\KeywordTok}[1]{\textcolor[rgb]{0.00,0.23,0.31}{#1}}
\newcommand{\NormalTok}[1]{\textcolor[rgb]{0.00,0.23,0.31}{#1}}
\newcommand{\OperatorTok}[1]{\textcolor[rgb]{0.37,0.37,0.37}{#1}}
\newcommand{\OtherTok}[1]{\textcolor[rgb]{0.00,0.23,0.31}{#1}}
\newcommand{\PreprocessorTok}[1]{\textcolor[rgb]{0.68,0.00,0.00}{#1}}
\newcommand{\RegionMarkerTok}[1]{\textcolor[rgb]{0.00,0.23,0.31}{#1}}
\newcommand{\SpecialCharTok}[1]{\textcolor[rgb]{0.37,0.37,0.37}{#1}}
\newcommand{\SpecialStringTok}[1]{\textcolor[rgb]{0.13,0.47,0.30}{#1}}
\newcommand{\StringTok}[1]{\textcolor[rgb]{0.13,0.47,0.30}{#1}}
\newcommand{\VariableTok}[1]{\textcolor[rgb]{0.07,0.07,0.07}{#1}}
\newcommand{\VerbatimStringTok}[1]{\textcolor[rgb]{0.13,0.47,0.30}{#1}}
\newcommand{\WarningTok}[1]{\textcolor[rgb]{0.37,0.37,0.37}{\textit{#1}}}

\providecommand{\tightlist}{%
  \setlength{\itemsep}{0pt}\setlength{\parskip}{0pt}}\usepackage{longtable,booktabs,array}
\usepackage{calc} % for calculating minipage widths
% Correct order of tables after \paragraph or \subparagraph
\usepackage{etoolbox}
\makeatletter
\patchcmd\longtable{\par}{\if@noskipsec\mbox{}\fi\par}{}{}
\makeatother
% Allow footnotes in longtable head/foot
\IfFileExists{footnotehyper.sty}{\usepackage{footnotehyper}}{\usepackage{footnote}}
\makesavenoteenv{longtable}
\usepackage{graphicx}
\makeatletter
\def\maxwidth{\ifdim\Gin@nat@width>\linewidth\linewidth\else\Gin@nat@width\fi}
\def\maxheight{\ifdim\Gin@nat@height>\textheight\textheight\else\Gin@nat@height\fi}
\makeatother
% Scale images if necessary, so that they will not overflow the page
% margins by default, and it is still possible to overwrite the defaults
% using explicit options in \includegraphics[width, height, ...]{}
\setkeys{Gin}{width=\maxwidth,height=\maxheight,keepaspectratio}
% Set default figure placement to htbp
\makeatletter
\def\fps@figure{htbp}
\makeatother

\KOMAoption{captions}{tableheading}
\makeatletter
\makeatother
\makeatletter
\makeatother
\makeatletter
\@ifpackageloaded{caption}{}{\usepackage{caption}}
\AtBeginDocument{%
\ifdefined\contentsname
  \renewcommand*\contentsname{Table of contents}
\else
  \newcommand\contentsname{Table of contents}
\fi
\ifdefined\listfigurename
  \renewcommand*\listfigurename{List of Figures}
\else
  \newcommand\listfigurename{List of Figures}
\fi
\ifdefined\listtablename
  \renewcommand*\listtablename{List of Tables}
\else
  \newcommand\listtablename{List of Tables}
\fi
\ifdefined\figurename
  \renewcommand*\figurename{Figure}
\else
  \newcommand\figurename{Figure}
\fi
\ifdefined\tablename
  \renewcommand*\tablename{Table}
\else
  \newcommand\tablename{Table}
\fi
}
\@ifpackageloaded{float}{}{\usepackage{float}}
\floatstyle{ruled}
\@ifundefined{c@chapter}{\newfloat{codelisting}{h}{lop}}{\newfloat{codelisting}{h}{lop}[chapter]}
\floatname{codelisting}{Listing}
\newcommand*\listoflistings{\listof{codelisting}{List of Listings}}
\makeatother
\makeatletter
\@ifpackageloaded{caption}{}{\usepackage{caption}}
\@ifpackageloaded{subcaption}{}{\usepackage{subcaption}}
\makeatother
\makeatletter
\@ifpackageloaded{tcolorbox}{}{\usepackage[skins,breakable]{tcolorbox}}
\makeatother
\makeatletter
\@ifundefined{shadecolor}{\definecolor{shadecolor}{rgb}{.97, .97, .97}}
\makeatother
\makeatletter
\makeatother
\makeatletter
\makeatother
\ifLuaTeX
  \usepackage{selnolig}  % disable illegal ligatures
\fi
\IfFileExists{bookmark.sty}{\usepackage{bookmark}}{\usepackage{hyperref}}
\IfFileExists{xurl.sty}{\usepackage{xurl}}{} % add URL line breaks if available
\urlstyle{same} % disable monospaced font for URLs
\hypersetup{
  pdftitle={Introdução às estruturas de dados e tipos de variáveis em R},
  pdfauthor={João Pedro Gonçalves Pacheco},
  colorlinks=true,
  linkcolor={blue},
  filecolor={Maroon},
  citecolor={Blue},
  urlcolor={Blue},
  pdfcreator={LaTeX via pandoc}}

\title{Introdução às estruturas de dados e tipos de variáveis em R}
\author{João Pedro Gonçalves Pacheco}
\date{}

\begin{document}
\maketitle
\ifdefined\Shaded\renewenvironment{Shaded}{\begin{tcolorbox}[enhanced, interior hidden, breakable, frame hidden, boxrule=0pt, borderline west={3pt}{0pt}{shadecolor}, sharp corners]}{\end{tcolorbox}}\fi

\renewcommand*\contentsname{Table of contents}
{
\hypersetup{linkcolor=}
\setcounter{tocdepth}{3}
\tableofcontents
}
Relembrando

\begin{description}
\item[objeto]
Um \textbf{objeto} é qualquer coisa que pode ser associado (e
armazenado) em uma variável.
\end{description}

As variáveis em R não são predefinidas; elas assumem o tipo (classe) do
objeto que for atribuído a elas.

Assim, uma instrução do tipo:

\texttt{x\ \textless{}-\ 1}

atribui a \emph{x} o valor (inteiro) 1, e é perfeitamente válido
reatribuir:

\texttt{x\ \textless{}-\ 2.5}

A classe define a forma que o objeto será manipulado pelas diferentes
funções.

\hypertarget{i.-classes-de-objetos-em-r}{%
\subsection{I. Classes de objetos em
R}\label{i.-classes-de-objetos-em-r}}

A classe de um objeto é muito importante dentro do R. É a partir dela
que as funções e operadores conseguem saber exatamente o que fazer com
um objeto.

Por exemplo, se tentarmos somar ``a'' e ``b'', uma mensagem de erro
aparecerá. Isso acontece porque o R sabe que ``a'' e ``b'' são objetos
do tipo texto.

\begin{Shaded}
\begin{Highlighting}[]
\DecValTok{1}\SpecialCharTok{+}\DecValTok{1}
\end{Highlighting}
\end{Shaded}

\begin{verbatim}
[1] 2
\end{verbatim}

\begin{Shaded}
\begin{Highlighting}[]
\StringTok{"a"} \SpecialCharTok{+} \StringTok{"b"}
\end{Highlighting}
\end{Shaded}

\begin{verbatim}
Error in "a" + "b": argumento não-numérico para operador binário
\end{verbatim}

\hypertarget{a.-numuxe9rico}{%
\subsubsection{A. Numérico}\label{a.-numuxe9rico}}

\hypertarget{inteiros}{%
\paragraph{1. Inteiros}\label{inteiros}}

Inteiros (\emph{integers}) são números inteiros sem nenhuma vírgula.

Em R, você pode criar um número inteiro usando a função
\texttt{as.integer()} ou simplesmente atribuindo um número inteiro a uma
variável.

\begin{Shaded}
\begin{Highlighting}[]
\CommentTok{\# Criando uma variável inteira}
\NormalTok{meu\_integer }\OtherTok{\textless{}{-}} \FunctionTok{as.integer}\NormalTok{(}\DecValTok{42}\NormalTok{)}

\CommentTok{\# Criando uma variável inteira}
\NormalTok{meu\_integer }\OtherTok{\textless{}{-}} \FunctionTok{as.integer}\NormalTok{(}\DecValTok{42}\NormalTok{)}
\end{Highlighting}
\end{Shaded}

\hypertarget{duplo-doublefloat}{%
\paragraph{\texorpdfstring{2. Duplo
(\emph{double/float})}{2. Duplo (double/float)}}\label{duplo-doublefloat}}

Double (também conhecido como \emph{float}) representa números com
pontos decimais.

Por padrão, R trata números com pontos decimais como duplos.

\begin{Shaded}
\begin{Highlighting}[]
\CommentTok{\# Criando uma variável double}
\NormalTok{meu\_double }\OtherTok{\textless{}{-}} \FloatTok{3.14}
\end{Highlighting}
\end{Shaded}

\hypertarget{b.-caractere-character}{%
\subsubsection{\texorpdfstring{B. Caractere
(\emph{character})}{B. Caractere (character)}}\label{b.-caractere-character}}

O tipo de dados \emph{Character} armazena texto ou \emph{strings}.

Os valores de texto são colocados entre aspas simples ou duplas.

\begin{Shaded}
\begin{Highlighting}[]
\CommentTok{\# Criando uma variável de caractere}
\NormalTok{meu\_caractere }\OtherTok{\textless{}{-}} \StringTok{"Olá, Mundo!"}
\end{Highlighting}
\end{Shaded}

\hypertarget{c.-luxf3gico-booleano}{%
\subsubsection{C. Lógico (Booleano)}\label{c.-luxf3gico-booleano}}

O tipo de dados lógicos representa valores binários, TRUE ou FALSE.

Os valores lógicos geralmente são o resultado de comparações ou
operações lógicas.

\begin{Shaded}
\begin{Highlighting}[]
\CommentTok{\# Criando uma variável lógica}
\NormalTok{está\_chovendo }\OtherTok{\textless{}{-}} \ConstantTok{TRUE}
\NormalTok{está\_ensolarado }\OtherTok{\textless{}{-}} \ConstantTok{FALSE}
\end{Highlighting}
\end{Shaded}

\hypertarget{d.-fator}{%
\subsubsection{D. Fator}\label{d.-fator}}

Fatores são usados para representar dados categóricos com níveis ou
categorias distintas.

Os fatores são úteis na análise estatística e na visualização de dados.

\begin{Shaded}
\begin{Highlighting}[]
\CommentTok{\# Criando uma variável de fator}
\NormalTok{escolaridade }\OtherTok{\textless{}{-}} \FunctionTok{factor}\NormalTok{(}\FunctionTok{c}\NormalTok{(}\StringTok{"Ensino Médio"}\NormalTok{, }\StringTok{"Faculdade"}\NormalTok{, }\StringTok{"Graduação"}\NormalTok{, }\StringTok{"Ensino Médio"}\NormalTok{, }\StringTok{"Graduação"}\NormalTok{))}
\end{Highlighting}
\end{Shaded}

\hypertarget{e.-data-e-hora}{%
\subsubsection{E. Data e Hora}\label{e.-data-e-hora}}

R tem classes específicas para lidar com dados de data e hora.

As classes mais comuns são Date, POSIXct e POSIXlt.

\begin{Shaded}
\begin{Highlighting}[]
\CommentTok{\# Criando uma variável de data}
\NormalTok{hoje }\OtherTok{\textless{}{-}} \FunctionTok{as.Date}\NormalTok{(}\StringTok{"2023{-}07{-}22"}\NormalTok{)}

\CommentTok{\# Criando uma variável POSIXct (data e hora com fuso horário)}
\NormalTok{current\_datetime }\OtherTok{\textless{}{-}} \FunctionTok{as.POSIXct}\NormalTok{(}\StringTok{"2023{-}07{-}22 14:30:00"}\NormalTok{, }\AttributeTok{tz =} \StringTok{"UTC"}\NormalTok{)}
\end{Highlighting}
\end{Shaded}

\hypertarget{ii.-estruturas-de-dados-buxe1sicas-em-r}{%
\subsection{II. Estruturas de dados básicas em
R}\label{ii.-estruturas-de-dados-buxe1sicas-em-r}}

\hypertarget{a.-vetores}{%
\subsubsection{A. Vetores}\label{a.-vetores}}

\hypertarget{criando-vetores}{%
\paragraph{Criando vetores}\label{criando-vetores}}

Os vetores são a estrutura de dados mais simples em R, representando uma
coleção de elementos do mesmo tipo de dados.

Você pode criar um vetor usando a função c(), que significa ``combinar''
ou ``concatenar''.

\begin{Shaded}
\begin{Highlighting}[]
\CommentTok{\# Criando um vetor numérico}
\NormalTok{vetor\_numerico }\OtherTok{\textless{}{-}} \FunctionTok{c}\NormalTok{(}\DecValTok{10}\NormalTok{, }\DecValTok{20}\NormalTok{, }\DecValTok{30}\NormalTok{, }\DecValTok{40}\NormalTok{, }\DecValTok{50}\NormalTok{)}

\CommentTok{\# Criando um vetor de caractere}
\NormalTok{vetor\_caractere }\OtherTok{\textless{}{-}} \FunctionTok{c}\NormalTok{(}\StringTok{"maçã"}\NormalTok{, }\StringTok{"banana"}\NormalTok{, }\StringTok{"laranja"}\NormalTok{)}

\CommentTok{\# Criando um vetor lógico}
\NormalTok{vetor\_logico }\OtherTok{\textless{}{-}} \FunctionTok{c}\NormalTok{(}\ConstantTok{TRUE}\NormalTok{, }\ConstantTok{FALSE}\NormalTok{, }\ConstantTok{TRUE}\NormalTok{)}
\end{Highlighting}
\end{Shaded}

\hypertarget{operauxe7uxf5es-buxe1sicas-subconjunto-aritmuxe9tica-luxf3gica}{%
\paragraph{Operações básicas (subconjunto, aritmética,
lógica)}\label{operauxe7uxf5es-buxe1sicas-subconjunto-aritmuxe9tica-luxf3gica}}

Subconjunto: Acessando elementos específicos de um vetor usando
indexação.

Aritmética: Realizar operações matemáticas em vetores (por exemplo,
adição, subtração, multiplicação).

Lógico: Executar operações lógicas em vetores (por exemplo, AND, OR).

\begin{Shaded}
\begin{Highlighting}[]
\CommentTok{\# Subconjunto}
\NormalTok{meu\_vetor }\OtherTok{\textless{}{-}} \FunctionTok{c}\NormalTok{(}\DecValTok{5}\NormalTok{, }\DecValTok{10}\NormalTok{, }\DecValTok{15}\NormalTok{, }\DecValTok{20}\NormalTok{, }\DecValTok{25}\NormalTok{)}
\NormalTok{element\_3 }\OtherTok{\textless{}{-}}\NormalTok{ meu\_vetor[}\DecValTok{3}\NormalTok{] }\CommentTok{\# Acesse o terceiro elemento (15)}

\CommentTok{\# Aritmética}
\NormalTok{vetor1 }\OtherTok{\textless{}{-}} \FunctionTok{c}\NormalTok{(}\DecValTok{1}\NormalTok{, }\DecValTok{2}\NormalTok{, }\DecValTok{3}\NormalTok{)}
\NormalTok{vetor2 }\OtherTok{\textless{}{-}} \FunctionTok{c}\NormalTok{(}\DecValTok{4}\NormalTok{, }\DecValTok{5}\NormalTok{, }\DecValTok{6}\NormalTok{)}
\NormalTok{soma\_vetor }\OtherTok{\textless{}{-}}\NormalTok{ vetor1 }\SpecialCharTok{+}\NormalTok{ vetor2 }\CommentTok{\# Adição elementar}

\CommentTok{\# Lógico}
\NormalTok{vetor\_logico }\OtherTok{\textless{}{-}} \FunctionTok{c}\NormalTok{(}\ConstantTok{TRUE}\NormalTok{, }\ConstantTok{FALSE}\NormalTok{, }\ConstantTok{TRUE}\NormalTok{)}
\NormalTok{all\_true }\OtherTok{\textless{}{-}} \FunctionTok{all}\NormalTok{(vetor\_logico) }\CommentTok{\# Verifica se todos os elementos são TRUE}
\end{Highlighting}
\end{Shaded}

\hypertarget{funuxe7uxf5es-vetoriais-comprimento-soma-muxe9dia-etc.}{%
\paragraph{Funções vetoriais (comprimento, soma, média,
etc.)}\label{funuxe7uxf5es-vetoriais-comprimento-soma-muxe9dia-etc.}}

R fornece várias funções para trabalhar com vetores, como calcular o
comprimento, soma, média, etc.

\begin{Shaded}
\begin{Highlighting}[]
\NormalTok{meu\_vetor }\OtherTok{\textless{}{-}} \FunctionTok{c}\NormalTok{(}\DecValTok{5}\NormalTok{, }\DecValTok{10}\NormalTok{, }\DecValTok{15}\NormalTok{, }\DecValTok{20}\NormalTok{, }\DecValTok{25}\NormalTok{)}

\CommentTok{\# Comprimento do vetor}
\NormalTok{comprimento\_vetor }\OtherTok{\textless{}{-}} \FunctionTok{length}\NormalTok{(meu\_vetor)}

\CommentTok{\# Soma e média do vetor}
\NormalTok{soma\_vetor }\OtherTok{\textless{}{-}} \FunctionTok{sum}\NormalTok{(meu\_vetor)}
\NormalTok{vector\_mean }\OtherTok{\textless{}{-}} \FunctionTok{mean}\NormalTok{(meu\_vetor)}
\end{Highlighting}
\end{Shaded}

\hypertarget{b.-matrizes}{%
\subsubsection{B. Matrizes}\label{b.-matrizes}}

Matrizes são estruturas de dados bidimensionais com linhas e colunas do
mesmo tipo de dados.

Você pode criar uma matriz usando a função matrix().

\begin{Shaded}
\begin{Highlighting}[]
\CommentTok{\# Criando uma matriz}
\NormalTok{my\_matrix }\OtherTok{\textless{}{-}} \FunctionTok{matrix}\NormalTok{(}\FunctionTok{c}\NormalTok{(}\DecValTok{1}\NormalTok{, }\DecValTok{2}\NormalTok{, }\DecValTok{3}\NormalTok{, }\DecValTok{4}\NormalTok{, }\DecValTok{5}\NormalTok{, }\DecValTok{6}\NormalTok{), }\AttributeTok{nrow =} \DecValTok{2}\NormalTok{, }\AttributeTok{ncol =} \DecValTok{3}\NormalTok{, }\AttributeTok{byrow =} \ConstantTok{TRUE}\NormalTok{)}

\CommentTok{\# Saída:}
\CommentTok{\# [,1] [,2] [,3]}
\CommentTok{\# [1,] 1 2 3}
\CommentTok{\# [2,] 4 5 6}
\end{Highlighting}
\end{Shaded}

Operações básicas (subconjunto, aritmética, multiplicação de matrizes)

\begin{itemize}
\item
  Subconjunto: Acessando elementos específicos, linhas ou colunas de uma
  matriz usando indexação.
\item
  Aritmética: Realização de operações aritméticas elementares em
  matrizes.
\item
  Multiplicação de matrizes: Realizando a multiplicação de matrizes
  usando o operador \%*\%.
\end{itemize}

Código de exemplo:

\begin{Shaded}
\begin{Highlighting}[]
\CommentTok{\# Subconfiguração}
\NormalTok{my\_matrix }\OtherTok{\textless{}{-}} \FunctionTok{matrix}\NormalTok{(}\FunctionTok{c}\NormalTok{(}\DecValTok{1}\NormalTok{, }\DecValTok{2}\NormalTok{, }\DecValTok{3}\NormalTok{, }\DecValTok{4}\NormalTok{, }\DecValTok{5}\NormalTok{, }\DecValTok{6}\NormalTok{), }\AttributeTok{nrow =} \DecValTok{2}\NormalTok{, }\AttributeTok{ncol =} \DecValTok{3}\NormalTok{, }\AttributeTok{byrow =} \ConstantTok{TRUE}\NormalTok{)}
\NormalTok{element\_2\_3 }\OtherTok{\textless{}{-}}\NormalTok{ my\_matrix[}\DecValTok{2}\NormalTok{, }\DecValTok{3}\NormalTok{] }\CommentTok{\# Acesse o elemento na segunda linha e terceira coluna (6)}

\CommentTok{\# Aritmética}
\NormalTok{matriz1 }\OtherTok{\textless{}{-}} \FunctionTok{matrix}\NormalTok{(}\FunctionTok{c}\NormalTok{(}\DecValTok{1}\NormalTok{, }\DecValTok{2}\NormalTok{, }\DecValTok{3}\NormalTok{, }\DecValTok{4}\NormalTok{), }\AttributeTok{nrow =} \DecValTok{2}\NormalTok{)}
\NormalTok{matriz2 }\OtherTok{\textless{}{-}} \FunctionTok{matrix}\NormalTok{(}\FunctionTok{c}\NormalTok{(}\DecValTok{5}\NormalTok{, }\DecValTok{6}\NormalTok{, }\DecValTok{7}\NormalTok{, }\DecValTok{8}\NormalTok{), }\AttributeTok{nrow =} \DecValTok{2}\NormalTok{)}
\NormalTok{soma\_matriz }\OtherTok{\textless{}{-}}\NormalTok{ matriz1 }\SpecialCharTok{+}\NormalTok{ matriz2 }\CommentTok{\# Adição elementar}

\CommentTok{\# Multiplicação da matriz}
\NormalTok{produto\_matriz }\OtherTok{\textless{}{-}}\NormalTok{ matriz1 }\SpecialCharTok{\%*\%}\NormalTok{ matriz2}
\end{Highlighting}
\end{Shaded}

Funções de matriz (dim, rowSums, colSums, etc.)

R fornece várias funções para trabalhar com matrizes, como obter
dimensões, somas de linhas, somas de colunas, etc.

\begin{Shaded}
\begin{Highlighting}[]
\NormalTok{minha\_matriz }\OtherTok{\textless{}{-}} \FunctionTok{matrix}\NormalTok{(}\FunctionTok{c}\NormalTok{(}\DecValTok{1}\NormalTok{, }\DecValTok{2}\NormalTok{, }\DecValTok{3}\NormalTok{, }\DecValTok{4}\NormalTok{, }\DecValTok{5}\NormalTok{, }\DecValTok{6}\NormalTok{), }\AttributeTok{nrow =} \DecValTok{2}\NormalTok{, }\AttributeTok{ncol =} \DecValTok{3}\NormalTok{, }\AttributeTok{byrow =} \ConstantTok{TRUE}\NormalTok{)}

\CommentTok{\# Dimensões da matriz}
\NormalTok{matrix\_dimensions }\OtherTok{\textless{}{-}} \FunctionTok{dim}\NormalTok{(minha\_matriz) }\CommentTok{\# Saída: 2 linhas e 3 colunas}

\CommentTok{\# Somas de linhas e somas de colunas}
\NormalTok{row\_sums }\OtherTok{\textless{}{-}} \FunctionTok{rowSums}\NormalTok{(minha\_matriz)}
\NormalTok{col\_sums }\OtherTok{\textless{}{-}} \FunctionTok{colSums}\NormalTok{(minha\_matriz)}
\end{Highlighting}
\end{Shaded}

\hypertarget{c.-sequuxeancias-arrays}{%
\subsubsection{\texorpdfstring{C. Sequências
(\emph{arrays})}{C. Sequências (arrays)}}\label{c.-sequuxeancias-arrays}}

Arrays são estruturas de dados multidimensionais que podem armazenar
elementos do mesmo tipo de dados.

Você pode criar um array usando a função array().

\begin{Shaded}
\begin{Highlighting}[]
\CommentTok{\# Criando um array}
\NormalTok{my\_array }\OtherTok{\textless{}{-}} \FunctionTok{array}\NormalTok{(}\FunctionTok{c}\NormalTok{(}\DecValTok{1}\NormalTok{, }\DecValTok{2}\NormalTok{, }\DecValTok{3}\NormalTok{, }\DecValTok{4}\NormalTok{, }\DecValTok{5}\NormalTok{, }\DecValTok{6}\NormalTok{), }\AttributeTok{dim =} \FunctionTok{c}\NormalTok{(}\DecValTok{2}\NormalTok{, }\DecValTok{3}\NormalTok{, }\DecValTok{2}\NormalTok{))}
\end{Highlighting}
\end{Shaded}

\hypertarget{d.-listas}{%
\subsubsection{D. Listas}\label{d.-listas}}

As listas são estruturas de dados versáteis que podem conter elementos
de diferentes tipos de dados.

Você pode criar uma lista usando a função list().

\begin{Shaded}
\begin{Highlighting}[]
\CommentTok{\# Criando uma lista}
\NormalTok{minha\_lista }\OtherTok{\textless{}{-}} \FunctionTok{list}\NormalTok{(}\DecValTok{1}\NormalTok{, }\StringTok{"olá"}\NormalTok{, }\ConstantTok{TRUE}\NormalTok{, }\FunctionTok{c}\NormalTok{(}\DecValTok{2}\NormalTok{, }\DecValTok{4}\NormalTok{, }\DecValTok{6}\NormalTok{))}
\end{Highlighting}
\end{Shaded}

Acessando elementos da lista

Você pode acessar elementos em uma lista usando indexação ou nomes (se
nomeados).

\begin{Shaded}
\begin{Highlighting}[]
\NormalTok{minha\_lista }\OtherTok{\textless{}{-}} \FunctionTok{list}\NormalTok{(}\DecValTok{1}\NormalTok{, }\StringTok{"olá"}\NormalTok{, }\ConstantTok{TRUE}\NormalTok{, }\FunctionTok{c}\NormalTok{(}\DecValTok{2}\NormalTok{, }\DecValTok{4}\NormalTok{, }\DecValTok{6}\NormalTok{))}

\CommentTok{\# Acessando elementos da lista}
\NormalTok{element\_1 }\OtherTok{\textless{}{-}}\NormalTok{ minha\_lista[[}\DecValTok{1}\NormalTok{]] }\CommentTok{\# Acesse o primeiro elemento (1)}
\NormalTok{element\_hello }\OtherTok{\textless{}{-}}\NormalTok{ minha\_lista[[}\DecValTok{2}\NormalTok{]] }\CommentTok{\# Acesse o segundo elemento}
\end{Highlighting}
\end{Shaded}

\hypertarget{e.-dataframes}{%
\subsubsection{E. Dataframes}\label{e.-dataframes}}

Os \emph{dataframes} são estruturas de dados bidimensionais que
armazenam dados em linhas e colunas.

Cada coluna em um quadro de dados pode ter um tipo de dados diferente.

Você pode criar um quadro de dados usando a função
\texttt{data.frame()}.

\begin{Shaded}
\begin{Highlighting}[]
\CommentTok{\# Criando um dataframe}
\NormalTok{dados\_aluno }\OtherTok{\textless{}{-}} \FunctionTok{data.frame}\NormalTok{(}
     \AttributeTok{nome =} \FunctionTok{c}\NormalTok{(}\StringTok{"Alice"}\NormalTok{, }\StringTok{"Bob"}\NormalTok{, }\StringTok{"Charlie"}\NormalTok{, }\StringTok{"David"}\NormalTok{),}
     \AttributeTok{idade =} \FunctionTok{c}\NormalTok{(}\DecValTok{22}\NormalTok{, }\DecValTok{21}\NormalTok{, }\DecValTok{23}\NormalTok{, }\DecValTok{20}\NormalTok{),}
\NormalTok{     pontuação }\OtherTok{=} \FunctionTok{c}\NormalTok{(}\DecValTok{85}\NormalTok{, }\DecValTok{78}\NormalTok{, }\DecValTok{92}\NormalTok{, }\DecValTok{80}\NormalTok{),}
     \AttributeTok{aprovado =} \FunctionTok{c}\NormalTok{(}\ConstantTok{TRUE}\NormalTok{, }\ConstantTok{FALSE}\NormalTok{, }\ConstantTok{TRUE}\NormalTok{, }\ConstantTok{TRUE}\NormalTok{)}
\NormalTok{)}
\end{Highlighting}
\end{Shaded}

\hypertarget{manipulando-dataframes-adicionando-removendo-renomeando-colunas}{%
\paragraph{Manipulando dataframes (adicionando, removendo, renomeando
colunas)}\label{manipulando-dataframes-adicionando-removendo-renomeando-colunas}}

Adicionar colunas: você pode adicionar novas colunas a um quadro de
dados usando o operador de atribuição (\$ ou {[}{]}).

Removendo colunas: Use a função subset() para excluir colunas
indesejadas.

Renomeando colunas: Use a função names() para alterar os nomes das
colunas.

\begin{Shaded}
\begin{Highlighting}[]
\CommentTok{\# Adicionando uma nova coluna}
\NormalTok{dados\_aluno}\SpecialCharTok{$}\NormalTok{genero }\OtherTok{\textless{}{-}} \FunctionTok{c}\NormalTok{(}\StringTok{"Feminino"}\NormalTok{, }\StringTok{"Masculino"}\NormalTok{, }\StringTok{"Masculino"}\NormalTok{, }\StringTok{"Masculino"}\NormalTok{)}

\CommentTok{\# Removendo uma coluna}
\NormalTok{dados\_aluno\_subset }\OtherTok{\textless{}{-}} \FunctionTok{subset}\NormalTok{(dados\_aluno, }\AttributeTok{select =} \SpecialCharTok{{-}}\NormalTok{idade)}

\CommentTok{\# Renomeando colunas}
\FunctionTok{names}\NormalTok{(dados\_aluno\_subset) }\OtherTok{\textless{}{-}} \FunctionTok{c}\NormalTok{(}\StringTok{"Nome"}\NormalTok{, }\StringTok{"Pontuação"}\NormalTok{, }\StringTok{"Aprovado"}\NormalTok{, }\StringTok{"Gênero"}\NormalTok{)}
\end{Highlighting}
\end{Shaded}

\hypertarget{operauxe7uxf5es-buxe1sicas-subconjunto-filtragem-classificauxe7uxe3o}{%
\paragraph{Operações básicas (subconjunto, filtragem,
classificação)}\label{operauxe7uxf5es-buxe1sicas-subconjunto-filtragem-classificauxe7uxe3o}}

\begin{itemize}
\item
  Subconjunto de linhas e colunas: você pode usar a indexação para
  acessar linhas ou colunas específicas em um quadro de dados.
\item
  Filtragem: Use condições lógicas para filtrar linhas com base em
  determinados critérios.
\item
  Classificação: Use a função order() para classificar o quadro de dados
  com base em uma coluna específica.
\end{itemize}

\begin{Shaded}
\begin{Highlighting}[]
\CommentTok{\# Subconjunto de linhas e colunas}
\CommentTok{\# Acesse as três primeiras linhas e as colunas "nome" e "pontuação"}
\NormalTok{subset\_data }\OtherTok{\textless{}{-}}\NormalTok{ dados\_aluno[}\DecValTok{1}\SpecialCharTok{:}\DecValTok{3}\NormalTok{, }\FunctionTok{c}\NormalTok{(}\StringTok{"nome"}\NormalTok{, }\StringTok{"pontuação"}\NormalTok{)]}

\CommentTok{\# Filtrando linhas}
\CommentTok{\# Filtre os alunos que passaram no exame}
\NormalTok{pass\_students }\OtherTok{\textless{}{-}}\NormalTok{ dados\_aluno[dados\_aluno}\SpecialCharTok{$}\NormalTok{passed }\SpecialCharTok{==} \ConstantTok{TRUE}\NormalTok{, ]}

\CommentTok{\# Classificando quadro de dados}
\CommentTok{\# Classifica o quadro de dados pela coluna "score" em ordem decrescente}
\NormalTok{sorted\_data }\OtherTok{\textless{}{-}}\NormalTok{ dados\_aluno[}\FunctionTok{order}\NormalTok{(}\SpecialCharTok{{-}}\NormalTok{dados\_aluno}\SpecialCharTok{$}\NormalTok{pontuação), ]}
\end{Highlighting}
\end{Shaded}

Observação: os dataframes são estruturas de dados fundamentais para
trabalhar com conjuntos de dados do mundo real em R. Eles permitem que
você armazene e manipule dados de forma eficaz para várias tarefas de
análise e visualização de dados. Os exemplos fornecidos aqui demonstram
algumas operações básicas, mas há muito mais funções e técnicas
disponíveis para manipulação de quadros de dados em R.

\hypertarget{iii.-coeruxe7uxe3o-e-conversuxe3o-de-tipo}{%
\subsection{III. Coerção e conversão de
tipo}\label{iii.-coeruxe7uxe3o-e-conversuxe3o-de-tipo}}

\hypertarget{a.-coeruxe7uxe3o-impluxedcita}{%
\subsubsection{A. Coerção
implícita}\label{a.-coeruxe7uxe3o-impluxedcita}}

A coerção implícita, também conhecida como coerção de tipo, ocorre
quando R converte automaticamente um tipo de dados em outro para
executar operações.

R segue regras específicas para coerção implícita para garantir que as
operações possam ser realizadas com sucesso.

\begin{Shaded}
\begin{Highlighting}[]
\CommentTok{\# Exemplos de coerção implícita}
\NormalTok{numeric\_vector }\OtherTok{\textless{}{-}} \FunctionTok{c}\NormalTok{(}\DecValTok{1}\NormalTok{, }\DecValTok{2}\NormalTok{, }\DecValTok{3}\NormalTok{, }\DecValTok{4}\NormalTok{)}
\NormalTok{character\_vector }\OtherTok{\textless{}{-}} \FunctionTok{c}\NormalTok{(}\StringTok{"5"}\NormalTok{, }\StringTok{"6"}\NormalTok{, }\StringTok{"7"}\NormalTok{, }\StringTok{"8"}\NormalTok{)}
\end{Highlighting}
\end{Shaded}

\hypertarget{b.-coeruxe7uxe3o-expluxedcita-usando-funuxe7uxf5es-as.numeric-as.character-as.logical-etc.}{%
\subsubsection{B. Coerção explícita usando funções (as.numeric,
as.character, as.logical,
etc.)}\label{b.-coeruxe7uxe3o-expluxedcita-usando-funuxe7uxf5es-as.numeric-as.character-as.logical-etc.}}

A coerção explícita permite que você converta explicitamente um tipo de
dados em outro usando funções de conversão específicas.

R fornece várias funções para conversão de tipo explícito, como
as.numeric(), as.character(), as.logical(), etc.

\begin{Shaded}
\begin{Highlighting}[]
\CommentTok{\# Exemplos de coerção explícita}
\NormalTok{numeric\_vector }\OtherTok{\textless{}{-}} \FunctionTok{c}\NormalTok{(}\DecValTok{1}\NormalTok{, }\DecValTok{2}\NormalTok{, }\DecValTok{3}\NormalTok{, }\DecValTok{4}\NormalTok{)}
\NormalTok{caractere\_vetor }\OtherTok{\textless{}{-}} \FunctionTok{c}\NormalTok{(}\StringTok{"5"}\NormalTok{, }\StringTok{"6"}\NormalTok{, }\StringTok{"7"}\NormalTok{, }\StringTok{"8"}\NormalTok{)}

\CommentTok{\# Converter character\_vector para numeric explicitamente}
\NormalTok{numeric\_character\_vector }\OtherTok{\textless{}{-}} \FunctionTok{as.numeric}\NormalTok{(caractere\_vetor)}

\CommentTok{\# Coerção de numérico para caractere}
\NormalTok{valor\_numérico }\OtherTok{\textless{}{-}} \DecValTok{42}
\NormalTok{character\_value }\OtherTok{\textless{}{-}} \FunctionTok{as.character}\NormalTok{(valor\_numérico)}

\CommentTok{\# Coerção para lógico (TRUE/FALSE)}
\NormalTok{vetor\_lógico }\OtherTok{\textless{}{-}} \FunctionTok{c}\NormalTok{(}\DecValTok{1}\NormalTok{, }\DecValTok{0}\NormalTok{, }\DecValTok{1}\NormalTok{, }\DecValTok{0}\NormalTok{)}
\NormalTok{resultado\_lógico }\OtherTok{\textless{}{-}} \FunctionTok{as.logical}\NormalTok{(vetor\_lógico)}
\end{Highlighting}
\end{Shaded}

Nota: Tenha cuidado com a conversão de tipo, pois pode resultar em
resultados inesperados se os dados não puderem ser convertidos com
precisão. Por exemplo, converter caracteres não numéricos em numéricos
pode resultar em NAs. Sempre valide os dados e certifique-se de que a
conversão pretendida é apropriada para sua análise.

A coerção explícita fornece mais controle sobre as conversões de tipo de
dados, permitindo que você assegure que os dados sejam transformados
adequadamente antes de uma análise mais aprofundada. A coerção implícita
pode ser útil em alguns casos, mas é essencial estar ciente de como o R
lida com as conversões de tipo para evitar possíveis problemas em seu
código.

\hypertarget{iv.-fatores-em-detalhe}{%
\subsection{IV. Fatores em detalhe}\label{iv.-fatores-em-detalhe}}

\hypertarget{a.-compreendendo-dados-categuxf3ricos}{%
\subsubsection{A. Compreendendo dados
categóricos}\label{a.-compreendendo-dados-categuxf3ricos}}

Os dados categóricos representam categorias ou grupos discretos e
distintos.

Os dados categóricos podem ser nominais (não ordenados) ou ordinais
(ordenados).

\begin{Shaded}
\begin{Highlighting}[]
\CommentTok{\# Dados categóricos nominais}
\NormalTok{eye\_colors }\OtherTok{\textless{}{-}} \FunctionTok{c}\NormalTok{(}\StringTok{"Azul"}\NormalTok{, }\StringTok{"Marrom"}\NormalTok{, }\StringTok{"Verde"}\NormalTok{, }\StringTok{"Verde"}\NormalTok{, }\StringTok{"Marrom"}\NormalTok{, }\StringTok{"Azul"}\NormalTok{)}

\CommentTok{\# Dados categóricos ordinais}
\NormalTok{education\_levels }\OtherTok{\textless{}{-}} \FunctionTok{c}\NormalTok{(}\StringTok{"Ensino Médio"}\NormalTok{, }\StringTok{"Faculdade"}\NormalTok{, }\StringTok{"Graduação"}\NormalTok{, }\StringTok{"Ensino Médio"}\NormalTok{, }\StringTok{"Graduação"}\NormalTok{)}
\end{Highlighting}
\end{Shaded}

\hypertarget{b.-criando-e-trabalhando-com-fatores}{%
\subsubsection{B. Criando e trabalhando com
fatores}\label{b.-criando-e-trabalhando-com-fatores}}

Fatores são usados para representar dados categóricos em R.

Você pode criar um fator usando a função factor().

\begin{Shaded}
\begin{Highlighting}[]
\CommentTok{\# Criando um fator a partir de dados categóricos nominais}
\NormalTok{eye\_colors\_factor }\OtherTok{\textless{}{-}} \FunctionTok{factor}\NormalTok{(eye\_colors)}

\CommentTok{\# Criando um fator a partir de dados categóricos ordinais}
\NormalTok{education\_levels\_factor }\OtherTok{\textless{}{-}} \FunctionTok{factor}\NormalTok{(education\_levels, }\AttributeTok{levels =} \FunctionTok{c}\NormalTok{(}\StringTok{"Ensino Médio"}\NormalTok{, }\StringTok{"Faculdade"}\NormalTok{, }\StringTok{"Graduação"}\NormalTok{), }\AttributeTok{ordered =} \ConstantTok{TRUE}\NormalTok{)}
\end{Highlighting}
\end{Shaded}

Trabalhando com fatores:

\texttt{levels()}: Obtenha os níveis/categorias únicos de um fator.

\texttt{nlevels()}: Obtenha o número de níveis/categorias em um fator.

\texttt{table()}: Cria uma tabela de frequência de níveis de fator.

\begin{Shaded}
\begin{Highlighting}[]
\CommentTok{\# Trabalhando com fatores}
\FunctionTok{levels}\NormalTok{(education\_levels\_factor) }\CommentTok{\# Output: "High School" "College" "Graduate"}
\end{Highlighting}
\end{Shaded}

\begin{verbatim}
[1] "Ensino Médio" "Faculdade"    "Graduação"   
\end{verbatim}

\begin{Shaded}
\begin{Highlighting}[]
\FunctionTok{nlevels}\NormalTok{(education\_levels\_factor) }\CommentTok{\# Saída: 3}
\end{Highlighting}
\end{Shaded}

\begin{verbatim}
[1] 3
\end{verbatim}

\begin{Shaded}
\begin{Highlighting}[]
\CommentTok{\# Criando uma tabela de frequência de níveis de fator}
\NormalTok{frequência\_tabela }\OtherTok{\textless{}{-}} \FunctionTok{table}\NormalTok{(eye\_colors\_factor)}

\CommentTok{\# Saída:}
\CommentTok{\# Azul Marrom Verde}
\CommentTok{\# 2 2 2}
\end{Highlighting}
\end{Shaded}

\hypertarget{c.-nuxedveis-de-fator-de-reordenauxe7uxe3o}{%
\subsubsection{C. Níveis de fator de
reordenação}\label{c.-nuxedveis-de-fator-de-reordenauxe7uxe3o}}

Em dados ordinais, a ordem dos níveis de fator é importante.

Você pode alterar a ordem dos níveis de fator usando a função factor()
com o argumento de níveis.

\begin{Shaded}
\begin{Highlighting}[]
\CommentTok{\# Reordenando os níveis dos fatores}
\CommentTok{\# Digamos que queremos os níveis em ordem crescente de escolaridade}
\NormalTok{education\_levels\_factor }\OtherTok{\textless{}{-}} \FunctionTok{factor}\NormalTok{(education\_levels, }\AttributeTok{levels =} \FunctionTok{c}\NormalTok{(}\StringTok{"Ensino Médio"}\NormalTok{, }\StringTok{"Faculdade"}\NormalTok{, }\StringTok{"Graduação"}\NormalTok{), }\AttributeTok{ordered =} \ConstantTok{TRUE}\NormalTok{)}

\CommentTok{\# Se quisermos os níveis em ordem decrescente de escolaridade}
\NormalTok{education\_levels\_factor }\OtherTok{\textless{}{-}} \FunctionTok{factor}\NormalTok{(education\_levels, }\AttributeTok{levels =} \FunctionTok{c}\NormalTok{(}\StringTok{"Graduação"}\NormalTok{, }\StringTok{"Faculdade"}\NormalTok{, }\StringTok{"Ensino Médio"}\NormalTok{), }\AttributeTok{ordered =} \ConstantTok{TRUE}\NormalTok{)}
\end{Highlighting}
\end{Shaded}

Você também pode usar a função reorder() para reordenar os níveis de
fator com base em uma variável ou critério específico.

\begin{Shaded}
\begin{Highlighting}[]
\CommentTok{\# Reordenando níveis de fator usando reorder()}
\NormalTok{pontuações }\OtherTok{\textless{}{-}} \FunctionTok{c}\NormalTok{(}\DecValTok{70}\NormalTok{, }\DecValTok{85}\NormalTok{, }\DecValTok{60}\NormalTok{, }\DecValTok{90}\NormalTok{)}
\NormalTok{education\_levels }\OtherTok{\textless{}{-}} \FunctionTok{c}\NormalTok{(}\StringTok{"Ensino Médio"}\NormalTok{, }\StringTok{"Faculdade"}\NormalTok{, }\StringTok{"Ensino Médio"}\NormalTok{, }\StringTok{"Graduação"}\NormalTok{)}

\CommentTok{\# Reordenar education\_levels com base nas pontuações}
\NormalTok{reordered\_education\_levels }\OtherTok{\textless{}{-}} \FunctionTok{reorder}\NormalTok{(education\_levels, pontuações, }\AttributeTok{FUN =}\NormalTok{ mean)}
\CommentTok{\# Use o fator reordenado em análise ou plotagem posterior}
\end{Highlighting}
\end{Shaded}

Fatores são essenciais para representar e analisar dados categóricos em
R. Compreender a distinção entre dados categóricos nominais e ordinais é
crucial para a modelagem e visualização de dados adequada. Ao trabalhar
com dados ordenados, reordenar os níveis de fator ajuda a manter a ordem
correta em análises e visualizações subsequentes.

\hypertarget{v.-pruxe1ticas-recomendadas-para-trabalhar-com-estruturas-de-dados}{%
\subsection{V. Práticas recomendadas para trabalhar com estruturas de
dados}\label{v.-pruxe1ticas-recomendadas-para-trabalhar-com-estruturas-de-dados}}

\hypertarget{a.-convenuxe7uxf5es-de-nomenclatura}{%
\subsubsection{A. Convenções de
nomenclatura}\label{a.-convenuxe7uxf5es-de-nomenclatura}}

Adotar convenções de nomenclatura consistentes e significativas para
estruturas de dados e variáveis é essencial para a legibilidade e
manutenção do código.

Use nomes descritivos que transmitam a finalidade e o conteúdo da
estrutura de dados.

Evite usar nomes que entrem em conflito com palavras reservadas ou
funções internas em R.

\begin{Shaded}
\begin{Highlighting}[]
\CommentTok{\# Boa convenção de nomenclatura}
\NormalTok{pontuação\_aluno }\OtherTok{\textless{}{-}} \FunctionTok{c}\NormalTok{(}\DecValTok{85}\NormalTok{, }\DecValTok{78}\NormalTok{, }\DecValTok{92}\NormalTok{, }\DecValTok{80}\NormalTok{)}
\NormalTok{average\_score }\OtherTok{\textless{}{-}} \FunctionTok{mean}\NormalTok{(pontuação\_aluno)}
\end{Highlighting}
\end{Shaded}

Evite usar nomes como ``média'' ou ``dados'' para suas variáveis

\hypertarget{b.-dados-organizados-tidy-data}{%
\subsubsection{B. Dados organizados (tidy
data)}\label{b.-dados-organizados-tidy-data}}

\hypertarget{o-que-suxe3o-dados-organizados}{%
\paragraph{O que são dados
organizados?}\label{o-que-suxe3o-dados-organizados}}

Tidy data é uma forma estruturada e padronizada de organizar dados,
popularizada por Hadley Wickham. Segue um conjunto de princípios que
simplificam a manipulação e análise de dados.

Em dados organizados, cada variável tem sua própria coluna, cada
observação tem sua própria linha e cada valor é colocado em uma célula.

Dados organizados:

\begin{itemize}
\item
  Promovem consistência e clareza
\item
  Agilizam os fluxos de trabalho de análise de dados
\item
  Tornam as tarefas de manipulação de dados mais diretas e intuitivas
\item
  Facilitam a integração perfeita com pacotes populares do R, como o
  alignr, o dplyr e o ggplot2.
\end{itemize}

\hypertarget{comparauxe7uxe3o-com-dados-confusos-ou-desarrumados}{%
\paragraph{Comparação com dados confusos ou
desarrumados}\label{comparauxe7uxe3o-com-dados-confusos-ou-desarrumados}}

Dados confusos carecem de uma estrutura padronizada e podem ter
variáveis espalhadas por várias colunas, observações misturadas ou
valores de dados colocados em células inadequadas.

\textbf{Princípios:}

\begin{enumerate}
\def\labelenumi{\arabic{enumi}.}
\item
  \textbf{Colunas são variáveis:} Em dados organizados, cada variável no
  conjunto de dados ocupa sua própria coluna. Isso permite uma
  identificação clara e fácil acesso a atributos de dados específicos.
\item
  \textbf{Linhas são observações:} Cada observação (ou ponto de dados
  individual) ocupa sua própria linha em dados organizados. Este arranjo
  permite a fácil identificação de casos ou amostras individuais.
\item
  \textbf{Os valores estão nas células:} Cada valor no conjunto de dados
  é colocado em uma célula específica, correspondente à sua variável e
  observação. Esse arranjo estruturado permite fácil recuperação e
  manipulação de dados.
\item
  \textbf{Um tipo de dados por tabela:} Cada tabela em dados organizados
  deve conter apenas um tipo de dados, como quantitativo, categórico ou
  textual. Essa separação clara simplifica a análise de dados e garante
  a integridade dos dados.
\end{enumerate}

\begin{Shaded}
\begin{Highlighting}[]
\CommentTok{\# Dados confusos}
\NormalTok{messy\_data }\OtherTok{\textless{}{-}} \FunctionTok{data.frame}\NormalTok{(}
   \AttributeTok{Nome =} \FunctionTok{c}\NormalTok{(}\StringTok{"Alice"}\NormalTok{, }\StringTok{"Bob"}\NormalTok{, }\StringTok{"Charlie"}\NormalTok{),}
   \AttributeTok{Math\_Score =} \FunctionTok{c}\NormalTok{(}\DecValTok{85}\NormalTok{, }\DecValTok{78}\NormalTok{, }\DecValTok{92}\NormalTok{),}
   \AttributeTok{Science\_Score =} \FunctionTok{c}\NormalTok{(}\DecValTok{90}\NormalTok{, }\DecValTok{88}\NormalTok{, }\DecValTok{76}\NormalTok{),}
   \AttributeTok{English\_Score =} \FunctionTok{c}\NormalTok{(}\DecValTok{80}\NormalTok{, }\DecValTok{94}\NormalTok{, }\DecValTok{88}\NormalTok{)}
\NormalTok{)}

\CommentTok{\# Dados organizados (as variáveis estão em colunas)}
\NormalTok{dados\_arrumados }\OtherTok{\textless{}{-}} \FunctionTok{data.frame}\NormalTok{(}
   \AttributeTok{Nome =} \FunctionTok{c}\NormalTok{(}\StringTok{"Alice"}\NormalTok{, }\StringTok{"Bob"}\NormalTok{, }\StringTok{"Charlie"}\NormalTok{),}
   \AttributeTok{Assunto =} \FunctionTok{c}\NormalTok{(}\StringTok{"Matemática"}\NormalTok{, }\StringTok{"Matemática"}\NormalTok{, }\StringTok{"Matemática"}\NormalTok{, }\StringTok{"Ciência"}\NormalTok{, }\StringTok{"Ciência"}\NormalTok{, }\StringTok{"Ciência"}\NormalTok{, }\StringTok{"Inglês"}\NormalTok{, }\StringTok{"Inglês"}\NormalTok{, }\StringTok{"Inglês"}\NormalTok{),}
\NormalTok{   Pontuação }\OtherTok{=} \FunctionTok{c}\NormalTok{(}\DecValTok{85}\NormalTok{, }\DecValTok{78}\NormalTok{, }\DecValTok{92}\NormalTok{, }\DecValTok{90}\NormalTok{, }\DecValTok{88}\NormalTok{, }\DecValTok{76}\NormalTok{, }\DecValTok{80}\NormalTok{, }\DecValTok{94}\NormalTok{, }\DecValTok{88}\NormalTok{)}
\NormalTok{)}
\end{Highlighting}
\end{Shaded}

\hypertarget{vi.-conclusuxe3o}{%
\subsection{VI. Conclusão}\label{vi.-conclusuxe3o}}

Hoje abordamos os conceitos fundamentais de estruturas de dados e tipos
de variáveis em R.

Os tipos de dados em R incluem numérico, caractere, lógico, fator e
data/hora.

Estruturas de dados básicas em R consistem em vetores, matrizes, arrays,
listas e dataframes.

A coerção e a conversão de tipo nos permitem converter dados entre
diferentes tipos de forma explícita e implícita.

Os fatores são essenciais para representar e analisar dados categóricos
em R, e aprendemos como criá-los e manipulá-los.

Exploramos as melhores práticas para trabalhar com estruturas de dados,
incluindo convenções de nomenclatura e tidy data.

\hypertarget{vii.-exercuxedcios-e-exemplos}{%
\subsection{VII. Exercícios e
Exemplos}\label{vii.-exercuxedcios-e-exemplos}}

\hypertarget{exercuxedcio-1}{%
\subsubsection{Exercício 1:}\label{exercuxedcio-1}}

Crie um vetor numérico contendo temperaturas em Celsius: 25, 30, 22, 28,
21.

Converta as temperaturas Celsius para Fahrenheit usando a fórmula:
Fahrenheit = (Celsius * 9/5) + 32.

Armazene o resultado em um novo vetor e imprima-o.

\begin{Shaded}
\begin{Highlighting}[]
\CommentTok{\# Exercício 1  }
\NormalTok{celsius\_temperatures }\OtherTok{\textless{}{-}} \FunctionTok{c}\NormalTok{(}\DecValTok{25}\NormalTok{, }\DecValTok{30}\NormalTok{, }\DecValTok{22}\NormalTok{, }\DecValTok{28}\NormalTok{, }\DecValTok{21}\NormalTok{)  }
\NormalTok{fahrenheit\_temperatures }\OtherTok{\textless{}{-}}\NormalTok{ (celsius\_temperatures }\SpecialCharTok{*} \DecValTok{9}\SpecialCharTok{/}\DecValTok{5}\NormalTok{) }\SpecialCharTok{+} \DecValTok{32}  
\FunctionTok{print}\NormalTok{(fahrenheit\_temperatures)}
\end{Highlighting}
\end{Shaded}

\begin{verbatim}
[1] 77.0 86.0 71.6 82.4 69.8
\end{verbatim}

\hypertarget{exercuxedcio-2}{%
\subsubsection{Exercício 2:}\label{exercuxedcio-2}}

Crie um vetor de caracteres representando os dias da semana:
``Segunda'', ``Terça'', \ldots, ``Domingo''.

Converta o vetor de caracteres em um fator com níveis ordenados
representando a ordem dos dias em uma semana.

Imprima o fator com os níveis na ordem correta.

\begin{Shaded}
\begin{Highlighting}[]
\CommentTok{\# Exercício 2  }
\NormalTok{dias\_da\_semana }\OtherTok{\textless{}{-}} \FunctionTok{c}\NormalTok{(}\StringTok{"segunda"}\NormalTok{, }\StringTok{"terça"}\NormalTok{, }\StringTok{"quarta"}\NormalTok{, }\StringTok{"quinta"}\NormalTok{, }\StringTok{"sexta"}\NormalTok{, }\StringTok{"sábado"}\NormalTok{, }\StringTok{"domingo"}\NormalTok{)  }
\NormalTok{ordenado\_dias }\OtherTok{\textless{}{-}} \FunctionTok{factor}\NormalTok{(dias\_da\_semana, }\AttributeTok{levels =} \FunctionTok{c}\NormalTok{(}\StringTok{"segunda"}\NormalTok{, }\StringTok{"terça"}\NormalTok{, }\StringTok{"quarta"}\NormalTok{, }\StringTok{"quinta"}\NormalTok{, }\StringTok{"sexta"}\NormalTok{, }\StringTok{"sábado"}\NormalTok{, }\StringTok{"domingo"}\NormalTok{), }\AttributeTok{ordered =} \ConstantTok{TRUE}\NormalTok{)  }
\FunctionTok{print}\NormalTok{(ordenado\_dias)}
\end{Highlighting}
\end{Shaded}

\begin{verbatim}
[1] segunda terça   quarta  quinta  sexta   sábado  domingo
Levels: segunda < terça < quarta < quinta < sexta < sábado < domingo
\end{verbatim}



\end{document}
